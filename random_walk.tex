\section{Random walks}
% by Bin Zhou, Wenjie Feng
% \begin{tabular}{ |c|c|c|  }
%  \hline
%  \multicolumn{3}{|c|}{Random on Graph } \\
%  \hline
%  & Iteration &Steady distribution  \\
%  \hline
%  Random   & $p_{t+1}=Wp_t$ &  $\frac{d}{1^Td} $\\
%   \hline
% Lazy random walk&  $p_{t+1}=\frac{1}{2}p_t +\frac{1}{2}Wp_t$ & $\frac{d}{1^Td} $\\
%  \hline
% Random walk with restart & $r_i =cWr_i+(1-c)e_i$& $ (1-c)(I-cW)^{-1}e_i$ \\
%  \hline
% Personal PageRank    & DZA&  012\\
%  \hline
% Spilling paint&   ASM&016\\

%  \hline
% \end{tabular}

\begin{itemize}
    \item Random walk \\
    Iteration:
    \[
        p^{t+1}=Wp^t
    \]
    Steady distribution 
    \[
    \frac{d}{1^Td}
    \]
    
    \item Lazy random walk \\
    Iteration:
    \[
       p^{t+1}=\frac{1}{2}p^t +\frac{1}{2}Wp^t
    \]
    Steady distribution 
    \[
    \frac{d}{1^Td}
    \]
    
     \item Random walk with restart \\
    Iteration:
    \[
       p_i^{t+1} =cWp_i^{t}+(1-c)e_i
    \]
    Steady distribution 
    \[
     (1-c)(I-cW)^{-1}e_i
    \]
    
     \item Personal PageRank  \\
    Iteration:
    \[
       p^{t+1}_u =\alpha \mathcal{X}_u+((1-\alpha)Wp^t)_u
    \]
    Steady distribution 
    \[
     (I-(1-\alpha)W)^{-1}\alpha \mathcal{X}_u
    \]
    $\mathcal{X}_u$ is a personal parameter for user $u$
     \item Spilling paint  \\
    Iteration:
    \[
    s^{t+1}=s^t+\alpha r^t
    \]
    \[
    r^{t+1}=(1-\alpha)Wr^t
    \]
    Steady distribution 
    \[
     \frac{2\alpha}{1+\alpha}(I-\frac{1-\alpha}{1+\alpha}W)^{-1}\mathcal{X}_u
    \]
    It is a diffusion process as diffusing paint in a graph. $\alpha$ is the fraction of the paint at each vertex dries in place.\\
    $s$ is the vector which records how much paint has become \textit{stuck} at each vertex.\\
    $r$ is the vector which indicates how much wet paint \textit{remain}s each vertex.
    
\end{itemize}


\subsection{Steady state}
Lazy random walk on a connected graph will converge to a stationary distribution $\pi$.\\
Random walk may not have steady state(For connected non-bipartite graph).\\
Random walk on a connected graph, and $t\ge 0$ 
\[
    ||p_t-\pi||\le \sqrt{\frac{\max_a d(a)}{\min_a d(a)}}\lambda_2^t
\]
If the walk starts at vertex $a$, for every vertex $b$ 
\[
 |p_t(b)-\pi(b)|\le \sqrt{\frac{d(b)}{d(a)}}\lambda_2^t
\]
\[
\lambda_2^t=(1-\mu)^t\le e^{-t\mu}
\]
note: $\lambda_1=1$, we focus on the gap between $\lambda_2$ and 1, so we write $\lambda_2=1-\mu$

\subsection{Random walk parameters}
\begin{itemize}
    \item Hitting time(Accessing time) \\
    The expected \# steps in a random walk before node $v$ is visited, starting from node $u$.
    \[
    H(u,v)=\mathbb{E}[\min\{ t\in \mathbb{N}\setminus \{0\}:X_t=v\}|X_0=u]
    \]
    \[
    H(u,u)=\frac{1}{\pi_u}=\frac{2|E|}{deg(u)}
    \]
    
    
    \item Commute time \\
    The expected \# steps in a random walk starting at $i$, before node $j$ is visited and then node $i$ is reached again.
    \[
    k(i,j)=H(i,j)+H(j,i)
    \]
    \[
    H(i,j)=\frac{1}{2}(k(i,j)+\sum_u \pi(u)[k(u,j)-k(u,i)])
    \]
    
    \item Covering time \\
    Expected time to visit all nodes.\\
    If start from a given node $u$:
\[
C(u)=\mathbb{E}[min{t\in \mathbb{N}\setminus \{0\}: \cup_{s=0}^t X_s=V}|X_0=u]
\]
If no starting node is specified:
\[
C=\max_{u\in V}C(u)
\]
\end{itemize}

\begin{theorem}
The hitting time between any two nodes of a graph on $n$ nodes is at most:
\begin{itemize}
    \item $\frac{4}{27}n^3-\frac{1}{9}n^2+\frac{2}{3}n -1$ if $n==0$ (mod 3)
    \item $\frac{4}{27}n^3-\frac{1}{9}n^2+\frac{2}{3}n -\frac{29}{27}$ if $n==1$ (mod 3)
    \item $\frac{4}{27}n^3-\frac{1}{9}n^2+\frac{4}{9}n -\frac{13}{27}$ if $n==2$ (mod 3)
\end{itemize}
\end{theorem}

\begin{theorem}
For the symmetric matrix $N=D^{-\frac{1}{2}}MD^{\frac{1}{2}}, M=D^{-1}A$\\
Eigen value-vector $(\lambda_i,v_i)$ of $N$. $\lambda_1\ge \lambda_2 \ge \cdots \ge \lambda_n$ \\
For any two vertices $s,t \in V$
\[
 H(s,t)=2|E|\sum_{k=2}^n\frac{1}{1-\lambda_k}(\frac{v_{k,t}^2}{deg(t)}-\frac{v_{k,s}v_{k,t}}{\sqrt{deg(s)deg(t)}})
\]
\end{theorem}

\begin{theorem}
\label{hitting_time}
For any vertex $s\in V$ 
\[
    \sum_{t\in V}\pi(t)H(s,t)=\sum_{k=2}^n\frac{1}{1-\lambda_k}
\]
\end{theorem}
The point behind theorem~\ref{hitting_time} is that the right hand side is independent of the starting vertex $s$

\begin{theorem}
\label{commute_time}
For any vertex $s,t \in V$
\[
k(s,t)=2m\sum_{k=2}^n\frac{1}{1-\lambda_k}(\frac{v_{kt}}{\sqrt{d(t)}}-\frac{v_{ks}}{\sqrt{d(s)}})^2
\]
using that $\frac{1}{2}\le \frac{1}{1-\lambda_k}\le \frac{1}{1-\lambda_2}$
\[
m(\frac{1}{d(s)}+\frac{1}{d(t)}) \le k(s,t) \le \frac{2m}{1-\lambda_2}(\frac{1}{d(s)}+\frac{1}{d(t)})
\]
\end{theorem}

\begin{theorem}
The cover time of any Graph $G$ is upper bounded by
\[
C\le 4(|V|-1)|E| \le 4n(n-1)^2
\]
\end{theorem}

\begin{theorem}
The cover time of any starting node in a graph with $n$ nodes:
\[
(1-o(1))n\log n \le C \le (\frac{4}{27}+o(1))n^3
\]
For any graph $G$ with $n=|V|$ vertices
\[
    n\ln n \lessapprox C \lessapprox n^3
\]
The cover time of regular graph on $n$ nodes 
\[
C\le 2n^2
\]
Locally computable transition probs. with
\[
    C\lessapprox n^2\log n
\]
\end{theorem}

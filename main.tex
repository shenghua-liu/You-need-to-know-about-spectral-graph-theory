%%%%%%%%%%%%%%%%%%%%%%%%%%%%%%%%%%%%%%%%%%%%%%%%%%%%%%%%%%%%%%%%%%%%%%
% writeLaTeX Example: A quick guide to LaTeX
%
% Source: Dave Richeson (divisbyzero.com), Dickinson College
% 
% A one-size-fits-all LaTeX cheat sheet. Kept to two pages, so it 
% can be printed (double-sided) on one piece of paper
% 
% Feel free to distribute this example, but please keep the referral
% to divisbyzero.com
% 
%%%%%%%%%%%%%%%%%%%%%%%%%%%%%%%%%%%%%%%%%%%%%%%%%%%%%%%%%%%%%%%%%%%%%%
% How to use writeLaTeX: 
%
% You edit the source code here on the left, and the preview on the
% right shows you the result within a few seconds.
%
% Bookmark this page and share the URL with your co-authors. They can
% edit at the same time!
%
% You can upload figures, bibliographies, custom classes and
% styles using the files menu.
%
% If you're new to LaTeX, the wikibook is a great place to start:
% http://en.wikibooks.org/wiki/LaTeX
%
%%%%%%%%%%%%%%%%%%%%%%%%%%%%%%%%%%%%%%%%%%%%%%%%%%%%%%%%%%%%%%%%%%%%%%

\documentclass[10pt,landscape]{article}
\usepackage{amssymb,amsmath,amsthm,amsfonts}
\usepackage{multicol,multirow}
\usepackage{calc}
\usepackage{ifthen}
\usepackage[landscape]{geometry}
\usepackage[colorlinks=true,citecolor=blue,linkcolor=blue]{hyperref}
\usepackage{algorithmic}
\usepackage{algorithm}
\usepackage{bm}

\ifthenelse{\lengthtest { \paperwidth = 11in}}
    { \geometry{top=.5in,left=.5in,right=.5in,bottom=.5in} }
	{\ifthenelse{ \lengthtest{ \paperwidth = 297mm}}
		{\geometry{top=1cm,left=1cm,right=1cm,bottom=1cm} }
		{\geometry{top=1cm,left=1cm,right=1cm,bottom=1cm} }
	}
\pagestyle{empty}
\makeatletter
\renewcommand{\section}{\@startsection{section}{1}{0mm}%
                                {-1ex plus -.5ex minus -.2ex}%
                                {0.5ex plus .2ex}%x
                                {\normalfont\large\bfseries}}
\renewcommand{\subsection}{\@startsection{subsection}{2}{0mm}%
                                {-1explus -.5ex minus -.2ex}%
                                {0.5ex plus .2ex}%
                                {\normalfont\normalsize\bfseries}}
\renewcommand{\subsubsection}{\@startsection{subsubsection}{3}{0mm}%
                                {-1ex plus -.5ex minus -.2ex}%
                                {1ex plus .2ex}%
                                {\normalfont\small\bfseries}}
\makeatother
\setcounter{secnumdepth}{0}
\setlength{\parindent}{0pt}
\setlength{\parskip}{0pt plus 0.5ex}
% -----------------------------------------------------------------------

\newtheorem{problem}{Problem}
\newtheorem{algo}{Algorithm}
\newtheorem{definition}{\textsc{Definition}}
%\newtheorem{proof}{Proof}
\newtheorem{axiom}{Axiom}
\newtheorem{observation}{Observation}
\newtheorem{trait}{Trait}
\newtheorem{iproblem}{Informal Problem}
\newtheorem{inference}{Inference}
\newtheorem{example}{\textbf{Example}}

\newcommand{\mb}{\mathbf}
\newcommand{\bit}{\begin{compactitem}}
	\newcommand{\eit}{\end{compactitem}}
\newcommand{\ben}{\begin{compactenum}}
	\newcommand{\een}{\end{compactenum}}

\newcommand{\bluecolor}{\textcolor{blue}}
\newcommand{\atn}[1]{\textcolor{red}{#1}}
\newcommand{\notice}[1]{{\textsf{\textcolor{green}{{\em [#1]}}}}}
\newcommand{\reminder}[1]{{\textsf{\textcolor{blue}{[#1]}}}}

\newcommand{\mkclean}{
	\renewcommand{\reminder}{\hide}
}

\newcommand{\SyncDual}{Synchronized Dual\xspace}
\newcommand{\syncdual}{synchronized dual\xspace}
\newcommand{\method}{\textsc{SyncDual}\xspace}

\newcommand{\tensor}[1]{\mathcal{#1}}   %% Tensor macro



\newcommand{\TX}{\tensor{X}}
\newcommand{\TR}{\tensor{R}}
\newcommand{\TB}{\tensor{B}}
\newcommand{\TS}{\tensor{S}}


\newcommand{\spot}{Spotlight\xspace}
\newcommand{\deltacon}{DeltaCon\xspace}
\newcommand{\wiscom}{WisCom\xspace}
\newcommand{\dcube}{\textsc{D-Cube}\xspace}
\newcommand{\mzoom}{\textsc{M-zoom}\xspace}
\newcommand{\sambaten}{SamBaTen\xspace}
\newcommand{\densestream}{DenseStream\xspace}
\newcommand{\densealert}{DenseAlert\xspace}
\newcommand{\cross}{\textsc{CrossSpot}\xspace}
\newcommand{\eigenspokes}{\textsc{EigenSpokes}\xspace}
\newcommand{\spoken}{\textsc{SpokEn}\xspace}
\newcommand{\oddball}{\textsc{OddBall}\xspace}
\newcommand{\fraudar}{\textsc{Fraudar}\xspace}

% Use Output in the format of Algorithm
\renewcommand{\algorithmicrequire}{\textbf{Input:}}
\renewcommand{\algorithmicensure}{\textbf{Output:}} 


\title{guide to Spectral Graph Theory}

\begin{document}

\raggedright
\footnotesize

\begin{center}
     \Large{\textbf{You need to know about Spectral Graph Theory}} \\
\end{center}
\begin{multicols}{3}
\setlength{\premulticols}{1pt}
\setlength{\postmulticols}{1pt}
\setlength{\multicolsep}{1pt}
\setlength{\columnsep}{2pt}

% ----------------------------------------
%my 
\newtheorem{theorem}{Theorem}
\newtheorem{corollary}{Corollary}[theorem]
\newtheorem{lemma}[theorem]{Lemma}

% ----------------------------------------
\section{Graph Laplacian}
% by Jiabao Zhang
\subsection{Important Formula}
\begin{itemize}
    \item $ L_G = D-A$
    \item $L_G = \sum_{e}L_e$
    \item
    \begin{aligned}
        L_G &= \sum_{(u,v)\in E}w_{u,v}(\delta_u - \delta_v)(\delta_u - \delta_v)^{T}\\
        &= \sum_{(u,v)\in E}w_{u,v}L_{G_{u,v}}
    \end{aligned}
    \item $x^{T}L_{G}x=\sum_{(u,v)\in E}w_{u,v}(x(u)-x(v))^2$
    \item If $\lambda_1 \leq \lambda_2 \leq \cdot \leq \lambda_n$
    \begin{itemize}
        \item $\lambda_2>0$. If and only if graph is connected.
        \item $\lambda_1=0$, its eigen vector $\phi_1=\textbf{1}$.
    \end{itemize}
    \item \textbf{Normalize Laplace}
    \[\displaystyle
        D^{-1/2}LD^{-1/2}=I - D^{-1/2}AD^{-1/2}
    \]
\end{itemize}

\subsection{Complete Graph}
Complete graph $K_n$, which has edge set $\{(u,v): u \neq v\}$.
\begin{lemma}
The Laplacian of $K_n$ has eigenvalue 0 with multiplicity 1 and $n$ with multiplicity $n-1$. 
\end{lemma}

\subsection{Star Graph}
Star graph $S_n$, which has edge set $\{(1,u): 2\leq u \leq n\}$.
\begin{lemma}
The Graph $S_n$ has eigenvalue 0 with multiplicity 1, eigenvalue 1 with multiplicity $n-2$, and $n$ with multiplicity 1. 
\end{lemma}

\subsection{HyperCube}
Graph with vertex set ${0,1}^d$, with edges between vertices whose names differ in exactly one bit. 
\begin{lemma}
The eigenvalues of its Laplacian matrix the numbers$(0, 2, ..., 2n)$, $k$-th eigenvalue has multiplicity $C_n^k$.
\end{lemma}
\subsection{Generalized Laplacian}
Graph $G$ with $n$ vertices, the generalized Laplacian $Q$ of G:
\begin{itemize}
    \item $Q_{uv} <0:(u,v)\in E$
    \item $Q_{uv} =0:(u,v)\notin E$
    \item $Q_{uu}$ Diagonal entries no constraints
\end{itemize}
\textbf{Not require} $Q\textbf{1}=0$
\section{Bounding Eigenvalues}
% Jiabao Zhang
The second smallest eigenvalue $\lambda_2$: related to minimum cut.
\subsection{Upper Bound}
\begin{theorem}[Courant-Fischer] Let $M$ be an $n\times n$ symmetric matrix with eigenvalues $\lambda_1 \leq \lambda_2 \leq \cdots \leq \lambda_n$ with corresponding eigenvectors $\phi_1, \cdots, \phi_n$. Then,
\begin{equation}
    \lambda_i = \min_{x \perp \phi_1,\cdots,\phi_{i-1}} \frac{x^{T}Mx}{x^{T}x}
\end{equation}
and the eigenvectors satisfy
\begin{equation}
    \phi_i = arg\min_{x \perp \phi_1,\cdots,\phi_{i-1}} \frac{x^{T}Mx}{x^{T}x}
\end{equation}
\end{theorem}
Recall
\begin{equation}
    \lambda_2 = \min_{v^T\textbf{1}=0}\frac{v^{T}Lv}{v^{T}v}
\end{equation}
So, every vector \textbf{v} orthogonal to \textbf{1} provides an upper bound on $\lambda_2$:
\begin{equation}
    \lambda_2 \leq \frac{v^{T}Lv}{v^{T}v}
\end{equation}

\subsection{Lower Bound}
\subsubsection{Graphic Inequality}
As $F \subseteq E$, we get
\begin{equation}
    L_G = \sum_{e \in E}L_e = \sum_{e \in F}L_e+\sum_{e \in E-F}L_e
    \succeq  \sum_{e \in F}L_e = L_H
\end{equation}

\begin{lemma}
If $G$ and $H$ are graphs such that $G \succeq c \cdot H$, then
\begin{equation}
    \lambda_{k}(G) \geq c\lambda_{k}(H),
\end{equation}
for all $k$.
\end{lemma}
\begin{proof}
The Courant-Fischer Theorem tells us that
\begin{equation}
\begin{aligned}
    \lambda_{k}(G)&=\min\limits_{dim(S)=k}\max\limits_{x\in S}\frac{x^{T}L_{G}x}{x^{T}x} \geq c \min\limits_{dim(S)=k} \max\limits_{x\in S}\frac{x^{T}L_{H}x}{x^{T}x}\\
    &=c\lambda_{k}(H).
\end{aligned}
\end{equation}
\end{proof}
\section{Graph Minimum/Maximum Cut}
% by Bin Zhou
\subsection{Definition}
For Graph $G(V,E,W)$, define capacity as:
\begin{equation}
\nonumber
Cap(A,B)=\sum_{(a,b)\in E,a\in A,b\in B}w(a,b)
\end{equation}
%\subsubsection{Quotient Cut}
\begin{itemize}
    \item \textbf{Quotient Cut}
    \[\displaystyle
    q(G)=\min_{S\subseteq V}\frac{Cap(S,\bar{S})}{\min\{|S|,|\bar{S}|\}}
    \]
    \item \textbf{Sparsest Cut}
    \[\displaystyle
    \alpha(G)=\min_{S\subseteq V}\frac{Cap(S,\bar{S})}{|S||\bar{S|}}
    \]
    Note:  $\frac{n}{2}\alpha(G)\leq q(G) \leq n\alpha (G)$
    \item \textbf{Conductance}
    \reminder{what is the definition of $Cap(S)$}
    \[\displaystyle
    \phi(G)=\min_{S\subseteq V}\frac{Cap(S,\bar{S})}{\min\{Cap(S), Cap(\bar{S})\}}
    \]
    Or given a mass at each vertex $m_i\geq d_i$, where $d_i$ is the degree of $v_i$. $Mass(V_i)=\sum_{v\in V_i } m_v$
    \[\displaystyle
    \phi(G)=\min_{S\subseteq V}\frac{Cap(S,\bar{S})}{\min\{Mass(S), Mass(\bar S)\}}
    \]
\end{itemize}


For $\lambda_1 \leq \lambda_2,...\leq \lambda_n$ is the eigen values of Laplacian of $G$
\begin{itemize}
    \item \textbf{Theorem 1}
    \[
    \frac{\lambda_2}{2}\leq q(G) \leq\sqrt{2\Delta \lambda_2}, \Delta=\max\{d_i\}
    \]
    \item \textbf{Theorem 2}
    \[
    \frac{\lambda_2}{2}\leq q(G) \leq\sqrt{2\Delta \lambda_2}, \Delta=\max\{d_i\}
    \]
\end{itemize}

\subsection{Graph Maximum Cut}
\textbf{Problem}
\[
Given: G=(V,E)
\]
\[
Goal: Max\quad \frac{1}{2}\sum_{i<j}w_{ij}(1-x_ix_j)
\]
\[
s.t. \quad x_i \in \{+1,-1\} \quad i=1...n
\]
\textbf{Define} 
$ALG(G)$ is the approximated solution found by our algorithm. $OPT(G)$ is the value of an optimal solution to max-cur on G

\textbf{Basic randomized algorithm}
\begin{algorithm}[H]
\caption{randomized algorithm}
\begin{algorithmic}[1]
\FOR{$v \in V$}
    \STATE pick t uniformly from [0,1)
    \IF{$t<0.5$}
        \STATE $v=-1$
    \ELSE
        \STATE $v=1$
     \ENDIF
\ENDFOR

\end{algorithmic}
\end{algorithm}
\textbf{0.529 - approximation algorithm}
\begin{algorithm}[H]
\caption{Trevisan’s algorithm }
\begin{algorithmic}[1]
\FOR{$v \in V$}
    \STATE compute $\beta = \lambda_{min}(I+D^{-1/2}AD^{-1/2})$
    \IF{$\beta>2\epsilon$}
        \RETURN Basic randomized algorithm
    \ELSE
        \STATE Compute $y$ statisfy $\beta=\frac{y^T(D+A)y}{y^Ty}$
        \STATE $x=D^{1/2}y$
        \STATE pick $t$ uniformly $[0,1]$
        \STATE $L=\{i|x(i)\geq \sqrt{t} \}$
        \STATE $R=\{i|x(i)\leq -\sqrt{t} \}$
        \STATE $S=L\cup R$
        \STATE recursive Max-Cut Alg$(V-S)=(L',R')$
        \RETURN \atn{better of $(L\cup L',R\cup R') or (L \cup R',R\cup L')$}
     \ENDIF
\ENDFOR
\end{algorithmic}
\label{alg:maxCut}
\end{algorithm}
For Alg.~\ref{alg:maxCut}, if we choose $\epsilon \approx 0.0554 $, then  $\frac{ALG(G)}{OPT(G)} \approx 0.529$

\textbf{Proof}

If $\beta \ge 2\epsilon$, then $OPT\le (1-\epsilon)|E|$, the basic randomized algorithm has \[
\mathbb{E}=\frac{1}{2}|E|\geq \frac{OPT}{2(1-\epsilon)}
\]
Note: Trevisan’s inequalities tell us 
\[
 \frac{1}{2}\beta_n \leq \beta(G) \leq \sqrt{2\beta_n}
\]
where $\beta_n$ is the smallest eigen value of $I+D^{-1/2}AD^{-1/2}$ 
Then 
\[
ALG(G)\geq |\delta(L,R)|+1/2\delta(S,V-S)+ALG(G-S)
\]
\[
OPT(G)\leq|E(L)|+|E(R)|+|\delta(L,R)|+|\delta(S,V-S)|+OPT(G-S)
\]
Then
\begin{equation}
\begin{gathered}
        \frac{ALG(G)}{OPT(G)}\geq\min\{X,\frac{ALG(G-S)}{OPT(G-S)}\} \\
    X=\frac{|\delta(L,R)|+\frac{1}{2}\delta(S,G-S)}{|E(L)|+|E(R)|+|\delta(L,R)|+|\delta(S,G-S)|}
\end{gathered}
\end{equation}

Since $\beta_n\leq 2\epsilon$, using Trevisan’s inequalities we bound:
\begin{equation}
    \begin{aligned}
    2\sqrt{\epsilon}&\geq \frac{2|E(L)|+2|E(R)|+|\delta(S,V-S)|}{vol(S)}\\
 &=\frac{2|E(L)|+2|E(R)|+|\delta(S,V-S)|}{2|E(L)|+2|E(R)|+|\delta(S,V-S)|+2|\delta(L,R)|}\\
 &=1-\frac{|\delta(L,R)|}{|E(L)|+|E(R)|+\frac{1}{2}|\delta(S,V-S)|+|\delta(L,R)|}
    \end{aligned}
\end{equation}
Thus 
\begin{equation}
    \begin{aligned}
    X&=
 \frac{|\delta(L,R)|+\frac{1}{2}\delta(S,V-S)}{|E(L)|+|E(R)|+|\delta(S,V-S)|+|\delta(L,R)|}\\
 &\geq
 \frac{|\delta(L,R)|}{|E(L)|+|E(R)|+\frac{1}{2}|\delta(S,V-S)|+|\delta(L,R)|} \\
 &\geq 1-2\sqrt{\epsilon}
    \end{aligned}
\end{equation}
So, we can conclude that
\begin{equation}
    \frac{ALG(G)}{OPT(G)}\geq \min\{1-s\sqrt{\epsilon}, \frac{ALG(G)}{OPT(G)}\}
\end{equation}
The same must hold true for $G-S$ recursively. But note that for some subgraph of $G$ we consider in some recursive step, it may be possible that $\beta_n \geq 2\epsilon$. Thus we conclude that:
\begin{equation}
        \frac{ALG(G)}{OPT(G)}\geq \min\{1-s\sqrt{\epsilon}, \frac{1}{2(1-\epsilon)}\}
\end{equation}
These two expressions are equal at $\epsilon= 0.0554$, at which point the ratio is about 0.529 



\section{Connected Components}

\subsection{Eigenvalue interlacing}
\begin{theorem}
Let $A$ be an n-by-n symmetric matrix and let $B$ be a principal submatrix of $A$ of dimension n-1(that is,$B$ is obtained by deleting the same row and column from $A$). Then,
\begin{equation}
    \alpha_1\geq\beta_1\geq \alpha_2\geq \beta_2\geq...\geq \alpha_{n-1}\geq \beta_{n-1}\geq \alpha_n
\end{equation}
where $\alpha_1 \geq \alpha_2\geq ... \geq \alpha_n$ and $\beta \geq \beta\geq ... \geq \beta_{n-1}$ are eigenvalues of $A$ and $B$, respectively.
\end{theorem}

\subsection{Perron-Frobenius Theorem}
\begin{theorem}
    Let $G$ be a connected graph, let $A$ be its adjacency matrix, and let $\mu_1 \geq \mu_2\geq... \geq \mu_n$ be its eigenvalues. Then
    \begin{itemize}
        \item $\mu_1 \geq -\mu_n $
        \item $\mu_1 > \mu_2$
        \item The eigenvalue $\mu_1$ has a strictly positive eigenvector.
    \end{itemize}
\end{theorem}
\begin{corollary}
\label{PFcorollary}
Let $M$ be a matrix with non-positive off-diagonal entries, such that the graph of non-zero off-diagonal entries is connected. Let $\lambda_1$ be the smallest eigenvalue of $M$ and let $v_1$ be the corresponding eigenvector. Then $v_1$ may be taken to be strictly positive, and $\lambda_1$ has multiplicity 1.
\end{corollary}

\begin{proof}
Consider the matrix $A=\sigma I -M$, for some large $\sigma$, this matrix will be non-negative, and the graph of its non-zero entries is connected. So, we may apply the Perron-Frobenius theory to $A$ to conclude that its largest eigenvalues $\alpha_1$ has multiplicity 11, and the corresponding eigenvector $v_1$ may be assumed to be strictly positive. We then have $\lambda_1 =\sigma - \alpha_1$, and $v_1$ is tan eigenvector of $\lambda_1$.
\end{proof}


\subsection{Fiedler’s Nodal Domain Theorem}
\begin{theorem}
    Let $G=(V,E,w)$ be a weighted connected graph, and let $L_G$ be its Laplacian matrix. Let $0=\lambda_1 <\lambda_2 \le ...\le \lambda_n$ be the eigenvalues of $L_G$ and let $v_1,...,v_n$ be the corresponding eigenvectors. Fo any $k\ge 2$, klet
    \[
        W_k=\{i\in V: v_k(i)\ge 0 \}.
    \]
    Then, the graph induced by $G$ on $W_k$ has at most k-1 connected components.
\end{theorem}
\begin{proof}
recall that $v_1=1$ and $v_k$ is orthogonal $v_1$. So $v_k $ has both positive and negative entries.\\
Assume $G(W_k)$ (Graph induced by $W_k$) has $t$ connected components. Re-order vertices let vertices in one conected component of $G(W_k)$ appear first.$(x_i\ge 0,y<0)$
\[
   \displaystyle \left[\begin{array}{ccccc}B_1 & 0 & 0& \cdots &c_1 \\0 & B_2 & 0& \cdots  &C_2 \\ \vdots & \vdots &\ddots  & \vdots & \vdots\\
   0& 0& \cdots  &B_t &C_t\\C_1^T &C_2^T&\cdots  &C_t^T&D \end{array}\right] \left(\begin{array}{c}x_1  \\x_2\\ \vdots \\x_t\\y \end{array}\right)= \lambda_k \left(\begin{array}{c}x_1  \\x_2\\ \vdots \\x_t\\y \end{array}\right)
\]
Graph of non-zero entries in each $B_i$ connected, and that each $C_i$ non-positive, and has at least one non-zero entry (otherwise the graph $G$ would be disconnected).\\
note: $y$ is the part of negative values in eigenvector $v_k$, D is the remaining elements of $L_G$ corresponding to $v_k$

As each entry in $C_i$ is non-positive and $y$ is strictly negative, each entry of $C_i y$ is non-negative and some entry of $C_i y$ is positive. Thus, $x_i$ cannot be all zeros,
\[
B_i x_i =\lambda_k x_i -C_i y \le \lambda_k x_i
\]
and 
\[
x_i^TB_i x_i \le \lambda_k x_i^T x_i
\]
\textbf{Goal}: prove $\lambda_{min}(B_i) <\lambda_k$
\begin{itemize}
    \item $x_i$ all positive,then $x_i C_i y >0$, and
    \[
        x_i^TB_i x_i = \lambda_k x_i^T x_i -x_i^TC_i y < \lambda_k x_i^T x_i
    \]
    \item $x_i$ has zero entry, Perron-Frobenius theorem tells us that $x_i$ cannot be an eigenvector of smallest eigenvalue, and so the smallest eigenvalue of $B_i $ less than $\lambda_k$

\end{itemize}
    note: refer to Corollary~\ref{PFcorollary},  $B_i$ has non-positive off-diagonal entries, and its non-zero off-diagonally entries are connected as the i-th connected component.
    
Thus, the matrix
\[
\displaystyle B=\left[\begin{array}{cccc}B_1 & 0 & \cdots &0 \\0 & B_2 &  \cdots  &0 \\ \vdots & \vdots &\ddots  & \vdots \\
   0& 0& \cdots  &B_t \end{array}\right]
\]

has  at least $t$ eigenvalues less than $\lambda_k$. By the eigenvalue interlacing theorem, this implies that $L_G$ has at least $t$ eigenvalues less than $\lambda_k$. We may conclude that $t$, the number of connected components of $G(W_k)$, is at most $k-1$.\\
e.g.
\[
    \lambda_1\le\lambda_1(B)\le \lambda_2 \le \lambda_2(B) \le \cdots \le \lambda_t(B) < \lambda_k \le \cdots \le \lambda_n
\]
So $t\le k-1$
\end{proof}
\section{Graph Sparsifier}
\subsection{Problem formulation }
\textbf{Input}
\[
G=(V,E,W), while \quad |E| \gg |V|
\]
\textbf{Output}
\[
H=(V,E',W'), while \quad |E| \approx |V| 
\]
Besides, $H$ is similar to $G$

\textbf{Similarity}
\begin{itemize}
    \item cut-preserving
    \[
    |\{w_{ij}|i\in S,J \notin S \}| \approx |\{w'_{ij}|i\in S,J \notin S \}|
    \]
    \item spectral-preserving (implies cut-preserving)
    \[
    (1-\epsilon)L_G \preceq L_H \preceq (1+\epsilon)L_G
    \]
\end{itemize}

\subsection{Graph as Resistor Network}
$L$ is the Laplacian of $G$, $\#node=n$, $\#edge=m$. \\
Let $w_{a,b}$ be the conductivity of edge $(a,b)$\\
\textbf{Define}: 
\[
    U_{m\times n}, U((a,b),c)=\begin{cases}1 & if a=c \\ -1& of b=c \\ 0 & otherwise \end{cases}
\]
\[
    W_{m\times m}, W((a,b),(c,d))=\begin{cases}w_{a,b} & if (a,b)=(c,d) \\  0 & otherwise \end{cases}
\]
Then the effective resistance $R_{eff}(a,b)$ from $a$ to $b$ is :
\[
    R_{eff}(a,b)=(\delta_a-\delta_b)^TL^+(\delta_a -\delta_b)
\]
note: $L^+$ is the pseudo-inverse of $L$, $\delta_i$ is a unit vector where i-th entry is equal to 1.
\subsection{Sparsifier algorithm}
\begin{algorithm}[H]
\caption{Sparsifier algorithm}
\begin{algorithmic}[1]
\STATE $q_{a,b}=w_{a,b}R_{eff}(a,b)$
\STATE $p_{a,b}=\min(1,C(\log n)\epsilon^{-2}q_{a,b})$
\FOR{$e \in E$}
    \STATE pick t uniformly from [0,1)
    \IF{$t<p_{e}$}
        \STATE include $e$ in $H$
        \STATE $w'_e =\frac{w_{e}}{p_e}$
     \ENDIF
\ENDFOR
\end{algorithmic}
\end{algorithm}
This sparsifier algorithm keeps the spectral-preserving, and the $\#edge$ in $H$ is $O(C n\log n\epsilon^{-2})$ 
\begin{proof} 

\begin{itemize}
    \item spectral-preserving
    \begin{equation*}
        \begin{aligned}
            E(L_H)&=\sum_{(a,b)\in E}p_{a,b}\frac{w_{a,b}}{p_{a,b}}L_{a,b}\\
            &=\sum_{(a,b)\in E}w_{a,b}L_{a,b} \\
            &=L_G
        \end{aligned}
    \end{equation*}
    \item $\#edge$ in $H$
     \begin{equation*}
        \begin{aligned}
           \sum_{(a,b)\in E}q_{a,b}&=\sum_{(a,b)\in E}w_{a,b}R_{eff}(a,b)\\
           &=\sum_{(a,b)\in E}w_{a,b}(\delta_a-\delta_b)^TL^+(\delta_a -\delta_b)\\
           &=\sum_{(a,b)\in E}w_{a,b}Tr(L^+(\delta_a-\delta_b)(\delta_a -\delta_b)^T)\\
           &=Tr(\sum_{(a,b)\in E}L^+w_{a,b}(\delta_a-\delta_b)(\delta_a -\delta_b)^T) \\
           &=Tr(L_G^+L_G)\\
           &=n-1
        \end{aligned}
    \end{equation*}
    So 
    \begin{equation*}
        \begin{aligned}
           \sum_{(a,b)\in E}p_{a,b}&=\sum_{(a,b)\in E}\min(1,C(\log n)q_{a,b}\epsilon^{-2})\\
           &\le \sum_{(a,b)\in E}C(\log n)q_{a,b}\epsilon^{-2}\\
            &= Cn \log n \epsilon^{-2} \sum_{(a,b)\in E}q_{a,b}\\
             &\le Cn \log n\epsilon^{-2}
        \end{aligned}
    \end{equation*}
    \item The effective resistances formation can use Chernoff bound, which promise: when number of edges in $H$ is larger, the probability decrease exponentially.

\end{itemize}
\end{proof}
\section{Power Iteration}
% Jiabao Zhang
\subsection{Define}
\begin{itemize}
    \item An approximation to \textbf{the dominant eigenvector}.
    \item Matrix $A$ can be decomposed into:
    \[
    A = VJV^{-1}
    \]
    \item Start vector $b_0$ can be written as:
    \[
    b_0 = c_1v_1+\cdots+c_nv_n
    \]
    \item Recurrence relation for $b_{k+1}$ is:
    \[
    b_{k+1}=\frac{Ab_{k}}{||Ab_{k}||}=\frac{A^{k+1}b_0}{||A^{k+1}b_0||}
    \]
\end{itemize}
\begin{proof}
\[
    \begin{aligned}
        b_k &= \frac{A^{k}b_0}{||A{k}b_0||}=\frac{(VJV^{-1})^{k}b_0}{||(VJV^{-1})^{k}b_0||}\\
        &=\frac{VJ^{k}V^{-1}(c_1v_1+\cdots+c_nv_n)}{||VJ^{k}V^{-1}(c_1v_1+\cdots+c_nv_n)||}\\
        &=\frac{VJ^{k}(c_1e_1+\cdots+c_ne_n)}{||VJ^{k}(c_1e_1+\cdots+c_ne_n)||}\\
        &= (\frac{\lambda_1}{|\lambda_1|})^{k}\frac{c_1}{|c_1|}\frac{v_1+\frac{1}{c_1}(\frac{1}{\lambda_1}J)^{k}(c_2e_2+\cdots+c_ne_n)}{||v_1+\frac{1}{c_1}(\frac{1}{\lambda_1}J)^{k}(c_2e_2+\cdots+c_ne_n)||}
    \end{aligned}
\]
The expression above simplifies as $k \to \infty$,
\[
\displaystyle (\frac{1}{\lambda_1}J)^{k} =\left[\begin{array}{cccc}[1] &  & & \\ & (\frac{1}{\lambda_1}J_2)^{k}  & &\\ &  &\ddots  &  \\
   & & & (\frac{1}{\lambda_1}J_m)^{k}\end{array}\right]
   \rightarrow \left[\begin{array}{cccc}1 &  & & \\ & 0 & &\\ &  &\ddots  &  \\
   & & & 0 \end{array}\right]
\]
So, 
\[
    b_k = (\frac{\lambda_1}{|\lambda_1|})^{k}\frac{c_1}{|c_1|}\frac{v_1}{||v_1||}
\]
\end{proof}
Note that $v_1$ is only a scalar, although the sequence $(b_k)$ may not converge, $b_k$ is nearly an eigenvector of $A$ for large $k$.
\textbf{Convergence rate} is:
\begin{equation}
    |\frac{\lambda_2}{\lambda_1}|
\end{equation}



\section{Random walks}
% by Bin Zhou, Wenjie Feng
% \begin{tabular}{ |c|c|c|  }
%  \hline
%  \multicolumn{3}{|c|}{Random on Graph } \\
%  \hline
%  & Iteration &Steady distribution  \\
%  \hline
%  Random   & $p_{t+1}=Wp_t$ &  $\frac{d}{1^Td} $\\
%   \hline
% Lazy random walk&  $p_{t+1}=\frac{1}{2}p_t +\frac{1}{2}Wp_t$ & $\frac{d}{1^Td} $\\
%  \hline
% Random walk with restart & $r_i =cWr_i+(1-c)e_i$& $ (1-c)(I-cW)^{-1}e_i$ \\
%  \hline
% Personal PageRank    & DZA&  012\\
%  \hline
% Spilling paint&   ASM&016\\

%  \hline
% \end{tabular}

\begin{itemize}
    \item Random walk \\
    Iteration:
    \[
        p^{t+1}=Wp^t
    \]
    Steady distribution 
    \[
    \frac{d}{1^Td}
    \]
    
    \item Lazy random walk \\
    Iteration:
    \[
       p^{t+1}=\frac{1}{2}p^t +\frac{1}{2}Wp^t
    \]
    Steady distribution 
    \[
    \frac{d}{1^Td}
    \]
    
     \item Random walk with restart \\
    Iteration:
    \[
       p_i^{t+1} =cWp_i^{t}+(1-c)e_i
    \]
    Steady distribution 
    \[
     (1-c)(I-cW)^{-1}e_i
    \]
    
     \item Personal PageRank  \\
    Iteration:
    \[
       p^{t+1}_u =\alpha \mathcal{X}_u+((1-\alpha)Wp^t)_u
    \]
    Steady distribution 
    \[
     (I-(1-\alpha)W)^{-1}\alpha \mathcal{X}_u
    \]
    $\mathcal{X}_u$ is a personal parameter for user $u$
     \item Spilling paint  \\
    Iteration:
    \[
    s^{t+1}=s^t+\alpha r^t
    \]
    \[
    r^{t+1}=(1-\alpha)Wr^t
    \]
    Steady distribution 
    \[
     \frac{2\alpha}{1+\alpha}(I-\frac{1-\alpha}{1+\alpha}W)^{-1}\mathcal{X}_u
    \]
    It is a diffusion process as diffusing paint in a graph. $\alpha$ is the fraction of the paint at each vertex dries in place.\\
    $s$ is the vector which records how much paint has become \textit{stuck} at each vertex.\\
    $r$ is the vector which indicates how much wet paint \textit{remain}s each vertex.
    
\end{itemize}


\subsection{Steady state}
Lazy random walk on a connected graph will converge to a stationary distribution $\pi$.\\
Random walk may not have steady state(For connected non-bipartite graph).\\
Random walk on a connected graph, and $t\ge 0$ 
\[
    ||p_t-\pi||\le \sqrt{\frac{\max_a d(a)}{\min_a d(a)}}\lambda_2^t
\]
If the walk starts at vertex $a$, for every vertex $b$ 
\[
 |p_t(b)-\pi(b)|\le \sqrt{\frac{d(b)}{d(a)}}\lambda_2^t
\]
\[
\lambda_2^t=(1-\mu)^t\le e^{-t\mu}
\]
note: $\lambda_1=1$, we focus on the gap between $\lambda_2$ and 1, so we write $\lambda_2=1-\mu$

\subsection{Random walk parameters}
\begin{itemize}
    \item Hitting time(Accessing time) \\
    The expected \# steps in a random walk before node $v$ is visited, starting from node $u$.
    \[
    H(u,v)=\mathbb{E}[\min\{ t\in \mathbb{N}\setminus \{0\}:X_t=v\}|X_0=u]
    \]
    \[
    H(u,u)=\frac{1}{\pi_u}=\frac{2|E|}{deg(u)}
    \]
    
    
    \item Commute time \\
    The expected \# steps in a random walk starting at $i$, before node $j$ is visited and then node $i$ is reached again.
    \[
    k(i,j)=H(i,j)+H(j,i)
    \]
    \[
    H(i,j)=\frac{1}{2}(k(i,j)+\sum_u \pi(u)[k(u,j)-k(u,i)])
    \]
    
    \item Covering time \\
    Expected time to visit all nodes.\\
    If start from a given node $u$:
\[
C(u)=\mathbb{E}[min{t\in \mathbb{N}\setminus \{0\}: \cup_{s=0}^t X_s=V}|X_0=u]
\]
If no starting node is specified:
\[
C=\max_{u\in V}C(u)
\]
\end{itemize}

\begin{theorem}
The hitting time between any two nodes of a graph on $n$ nodes is at most:
\begin{itemize}
    \item $\frac{4}{27}n^3-\frac{1}{9}n^2+\frac{2}{3}n -1$ if $n==0$ (mod 3)
    \item $\frac{4}{27}n^3-\frac{1}{9}n^2+\frac{2}{3}n -\frac{29}{27}$ if $n==1$ (mod 3)
    \item $\frac{4}{27}n^3-\frac{1}{9}n^2+\frac{4}{9}n -\frac{13}{27}$ if $n==2$ (mod 3)
\end{itemize}
\end{theorem}

\begin{theorem}
For the symmetric matrix $N=D^{-\frac{1}{2}}MD^{\frac{1}{2}}, M=D^{-1}A$\\
Eigen value-vector $(\lambda_i,v_i)$ of $N$. $\lambda_1\ge \lambda_2 \ge \cdots \ge \lambda_n$ \\
For any two vertices $s,t \in V$
\[
 H(s,t)=2|E|\sum_{k=2}^n\frac{1}{1-\lambda_k}(\frac{v_{k,t}^2}{deg(t)}-\frac{v_{k,s}v_{k,t}}{\sqrt{deg(s)deg(t)}})
\]
\end{theorem}

\begin{theorem}
\label{hitting_time}
For any vertex $s\in V$ 
\[
    \sum_{t\in V}\pi(t)H(s,t)=\sum_{k=2}^n\frac{1}{1-\lambda_k}
\]
\end{theorem}
The point behind theorem~\ref{hitting_time} is that the right hand side is independent of the starting vertex $s$

\begin{theorem}
\label{commute_time}
For any vertex $s,t \in V$
\[
k(s,t)=2m\sum_{k=2}^n\frac{1}{1-\lambda_k}(\frac{v_{kt}}{\sqrt{d(t)}}-\frac{v_{ks}}{\sqrt{d(s)}})^2
\]
using that $\frac{1}{2}\le \frac{1}{1-\lambda_k}\le \frac{1}{1-\lambda_2}$
\[
m(\frac{1}{d(s)}+\frac{1}{d(t)}) \le k(s,t) \le \frac{2m}{1-\lambda_2}(\frac{1}{d(s)}+\frac{1}{d(t)})
\]
\end{theorem}

\begin{theorem}
The cover time of any Graph $G$ is upper bounded by
\[
C\le 4(|V|-1)|E| \le 4n(n-1)^2
\]
\end{theorem}

\begin{theorem}
The cover time of any starting node in a graph with $n$ nodes:
\[
(1-o(1))n\log n \le C \le (\frac{4}{27}+o(1))n^3
\]
For any graph $G$ with $n=|V|$ vertices
\[
    n\ln n \lessapprox C \lessapprox n^3
\]
The cover time of regular graph on $n$ nodes 
\[
C\le 2n^2
\]
Locally computable transition probs. with
\[
    C\lessapprox n^2\log n
\]
\end{theorem}

\section{Random Spanning Tree}
\subsection{Define}
Let $G(V, E)$ be a finite graph and $n = |V|$. The following three conditions are equivalent:
\begin{itemize}
    \item G is a tree.
    \item G is circuit-free and has $n-1$ edges.
    \item G is connected and has $n-1$ edges.
\end{itemize}
\subsection{Count trees}
Let us define \textbf{in-degree matrix D} of a digraph $G(V, E)$, where $V=\{1, 2, \cdots, n\}$, as follows:
\[
    D(i, j)=\begin{cases}d_{in}(i) & if\ i=j \\ -k& if \ i \neq j, where\ k \ is \ the \ number \ of \ edges \ from \ i \ to \ j. \end{cases}
\]
\begin{theorem}
The number of directed spanning trees with root $r$ of a digraph the minor of its \emph{in-degree matrix} which results from the erasure of the $r$th row and column.
\end{theorem}

\subsection{Random Spanning Tree}
All the directed spanning trees of $G_M$ rooted at $i$ is denoted $T_{i}(G_M)$ and all rooted directed spanning trees of $G_M$ denoted $T(G_M)$.
\begin{theorem}[Markov chain tree theorem]
Let $M$ be an irreducible Markov chain on $n$ states with stationary distribution $\pi_1, \cdots, \pi_n$. Let $G_M$ be the directed graph associated with $M$. Then
\begin{equation}
    \pi_{i} = \frac{\sum_{T \in \mathcal{T}_i(G_M)}w(T)}{\sum_{T \in \mathcal{T}(G_M)}w(T)}, w(T) = \prod_{e \in T}{w(e)}.
\end{equation}
\end{theorem}


\section{Maximum Flow: Energy Flow}
% by Bin Zhou
Current fastest algorithm $O(n^{3/2})$(Lee Sidford '14)\\
\textbf{Our focus}: Sparse graph $(m=O(n))$ and unit-capacity(U=1) regime.
Find $(1-\epsilon)$-approx. $O(n^{4/3}\epsilon^{-3})$-time algorithm.

\subsection{Finding electrical flows}
Let $L$ be the laplacian for graph $G(V,E)$ where edge $e$ has weight $\frac{1}{R_e }$.  The potential vector $\phi$ for the electric flow and the current vector $i$ whose j-th
component is the outgoing current at vertex $j$ are related as follows,
\begin{equation}
\begin{aligned}
    L\phi=i \\
    L^+i=\phi
    \end{aligned}
\end{equation}
The Laplacian linear system solver can be used to find the electrical flow for any $i$ in time $O(m)$ by finding $\phi$ and then the current through each edge using Ohm's law.

\subsection{Algorithm for max flow}
\textbf{Binary search:} Suppose we have an algorithm that constructs a flow of value $f$ if one exists. The value of the max flow $F^*$ can be found in $\log n$ iterations by binary search and the max flow can be found by running the algorithm with input $F^*$.
\begin{algorithm}[H]
\caption{Algorithm for max flow}
\begin{algorithmic}[1]
\STATE $LB=0$,$UB=n^2$

\WHILE{$UB-LB>\epsilon$}
    \STATE  $F^*=\frac{UB+LB}{2}$
    \STATE run the algorithm for $F^*$
    \IF{exist flow}
        \STATE $LB=F^*$
    \ELSE
        \STATE $UB=F^*$
     \ENDIF
\ENDWHILE
\end{algorithmic}
\end{algorithm}

\begin{algorithm}[H]
\caption{Algorithm for $F^*$}
\begin{algorithmic}[1]
\STATE Treat edges as resistors of resistance $r_e=1$
\STATE $i=0$
\WHILE{$i < O(\rho \epsilon^{-2})$}
    \STATE Compute electrical flow $f$ of value $F^*$
    \STATE $\rho_i=\max_e|f^i(e)|$
    \STATE $\rho=\max(\rho,\rho_i)$
    \FOR{$e \in E$}
        \IF{$f^i(e)>O(n^{1/3})$}
            \STATE remove $e$
        \ELSE
            \STATE $r^i_e=r_e^{i-1}(1+\epsilon|f^i(e)|\rho^{-1}_i)$
            \STATE $i=i+1$
        \ENDIF
    \ENDFOR
\ENDWHILE


\end{algorithmic}
\end{algorithm}
\section{Regression Problem: Solving Linear System}
\begin{table}[htbp]
    \centering
    \begin{tabular}{|c|c|c|}
         Method & Convergence rate & Iteration \\
         \hline
         Cheybshev & $\frac{1}{K(B^{-1}A)}$ & $\sqrt{\frac{\lambda_{n}}{\lambda_{1}}}\ln\epsilon^{-1}$\\
         \hline
         Conjugate Gradient & $\frac{1}{\sqrt{K(B^{-1}A)}}$ & $O(m^{1/2}\log _n)$\\
         \hline
    \end{tabular}
    \caption{Caption}
    \label{tab:my_label}
\end{table}
\textbf{Goal:} solve $Ax = b$. ($A$ is spd)
\subsection{Precondition}
Find a matrix $B$ that is approximates $A$, and solve:
\[
    B^{-1}Ax = B^{-1}b
\]
where $B$ is called a preconditioner, then we get iterative method:
\[
    x^{n+1} = x^{n} + B^{-1}(b - Ax^{n}),
\]
\[
    x^{0} = B^{-1}b
\]
Now, we see that $\bm{B^{-1}b}$ is a good approximate of $\bm{x}$ in the $A$-norm.
\begin{lemma}[Precondition Approximation]
Let $A$ and $B$ be positive definite matrices such that
\[
    \alpha B \preceq A \preceq \beta B.
\]
Then all the eigenvalues of $\bm{B^{−1}A}$ lie between $\alpha$ and $\beta$.
\begin{proof}
We will just prove the upper bound. We have
\begin{equation*}
    \begin{aligned}
        \lambda_{max}(B^{−1}A) &=  \lambda_{max}(B^{−1/2}A^{B−1/2})\\
        &= \max_{x} \frac{x^T B^{−1/2}AB^{−1/2}x}{x^{T}x}  \\
        &= \max_{y}\frac{y^{T}Ay}{y^{T}By}  ~~~~~~~~~~~ setting \ y=B^{−1/2}x, \\
        & \leq \beta.
    \end{aligned}
\end{equation*}
\end{proof}
\end{lemma}
So, if $B$ is an $\epsilon$-approximation of $A$ then all of the eigenvalues of
\begin{equation}
    I − AB^{−1}
\end{equation}
have absolute value at most $\epsilon$.
Then, we can conduct that
\begin{equation*}
    \begin{aligned}
        ||B^{-1}b−x||_{A}&=||A^{1/2}B^{-1}b - A^{1/2}x|| \\
        &= ||A^{1/2}B^{-1}Ax - A^{1/2}x|| \\
        &= ||A^{1/2}B^{-1}A^{1/2} - I|| ||A^{1/2}x|| \\
        & \leq \epsilon ||A^{1/2}x||= \epsilon ||x||_{A}
    \end{aligned}
\end{equation*}

\subsection{Preconditioned Cheybshev}
Let $q_{t}$ be a polynomial that is 1 at 0 and that has absolute value less than $\epsilon$ at each of the eigenvalues $\lambda_{i}$, and let pt be the polynomial such that $q_{t} = 1 − xp_{t}(x)$. To use this polynomial to solve the system, we set
\[
    x_{t} = p_{t}(B^{−1}A)B^{−1}b.
\]

\subsection{Preconditioned Conjugate Gradient}

\begin{itemize}
    \item Idea: same as Conjugate Gradient method, which finds $x_{t}$ in 
	Krylov subspace of $\{b, Ab, \cdots, A^{t}b\}$.
	\item Krylov subspace:
	\[
	    \{B^{-1}b, (B^{-1}A)B^{-1}b, \cdots, (B^{-1}A)^{t}B^{-1}b\}
	\]
	\item Algorithm: find the vector $\bm{x_{t}}$ that minimizes the error in the $A$-norm.
\end{itemize}

\subsection{Subgraph Preconditioning}
If $H$ is a subgraph of $G$, then
\[
    L_H \preceq L_G
\]
\[
    \lambda(L_H^{-1}L_G) \geq 1
\]
Thus,
\[
    K(L_H^{-1}L_G) \leq \lambda_{max}(L_H^{-1}L_G)
\]
for any subgraph preconditioner.

\subsubsection{Low stretch spanning tree}
For any edge $e \in E$, we now define the stretch of $e$ with respect to $T$. let $e_1, \cdots, c_k \in F$ be the edges on the unique path in $T$ connecting the endpoints of $e$. Then, the stretch of $e$ with respect to $T$ is given by 
\[
    st_{T}(e)=w(e)(\sum_{i=1}^{k}\frac{1}{w(e_{i})})
\]
which the stretch of graph $G$ is
\[
    st_{T}(G) = \sum_{e \in E}st_{T}(e)
\]
\begin{theorem}
Every weighted graph $G$ has a spanning tree subgraph $T$ such that the sum of the stretches of all edges of $G$ with respect to $T$ is at most
\[
    O(m \log n \log\log n),
\]
where $m$ is the number of edges $G$. Moreover, one can compute this tree in time $O(m \log n \log\log n)$.
Thus, if we choose a low-stretch spanning tree $T$, we will ensure that
\begin{equation*}
    \begin{aligned}
        Tr(L_T^{-1}L_G) &\leq \sum_{(u,v)\in E}w_{u,v}(\mathcal{X}_{u} - \mathcal{X}_{v})^{T}L_{T}^{-1}(\mathcal{X}_{u} - \mathcal{X}_{v}) \\
        &\leq O(m \log n \log\log n).
    \end{aligned}
\end{equation*}
    
\end{theorem}
So, 
\[
    \lambda_{max}(L_T^{-1}L_G) \leq O(m \log n \log\log n).
\]
PCG will require at most $O(m^{1/2} \log n)$ iterations, each of which requires one multiplication by $L_G$ and one linear solve in $L_T$ .
\section{Links}
    Papers on spectral graph theory: \cite{spielman2007spectral}

\bibliographystyle{abbrv}
\bibliography{sgtref}
% \section{What is \LaTeX?}
\LaTeX (usually pronounced ``LAY teck,'' sometimes ``LAH teck,'' and never ``LAY tex'') is a mathematics typesetting program that is the standard for most professional mathematics writing. It is based on the typesetting program \TeX\ created by Donald Knuth of Stanford University (his first version appeared in 1978). Leslie Lamport was responsible for creating \LaTeX\, a more user friendly version of \TeX. A team of \LaTeX\ programmers created the current version,  \LaTeX\ 2$\varepsilon$.

\section{Math vs. text vs. functions}
In properly typeset mathematics  variables appear in italics (e.g., $f(x)=x^{2}+2x-3$). The exception to this rule is predefined functions (e.g., $\sin (x)$). Thus it is important to \textbf{always} treat text, variables, and functions correctly. See the difference between $x$ and x, -1 and $-1$, and $sin(x)$ and $\sin(x)$.  

There are two ways to present a mathematical expression--- \emph{inline} or as an \emph{equation}.

\subsection{Inline mathematical expressions}
Inline expressions occur in the middle of a sentence.  To produce an inline expression, place the math expression between dollar signs (\verb!$!).  For example, typing \verb!$90^{\circ}$ is the same as $\frac{\pi}{2}$ radians!  yields $90^{\circ}$ is the same as $\frac{\pi}{2}$ radians.

\subsection{Equations}
Equations are mathematical expressions that are given their own line and are centered on the page.  These are usually used for important equations that deserve to be showcased on their own line or for large equations that cannot fit inline. To produce an inline expression, place the mathematical expression  between the symbols  \verb!\[! and \verb!\]!. Typing \verb!\[x=\frac{-b\pm\sqrt{b^2-4ac}}{2a}\]! yields \[x=\frac{-b\pm\sqrt{b^2-4ac}}{2a}.\]
 
\subsection{Displaystyle} 
To get full-sized inline mathematical expressions  use  \verb!\displaystyle!. Use this sparingly. Typing \verb!I want this $\displaystyle \sum_{n=1}^{\infty}! \verb!\frac{1}{n}$, not this $\sum_{n=1}^{\infty}! \verb!\frac{1}{n}$.! yields\\ I want  this $\displaystyle \sum_{n=1}^{\infty}\frac{1}{n}$, not this $\sum_{n=1}^{\infty}\frac{1}{n}.$


\section{Images}

You can put images (pdf, png, jpg, or gif) in your document. They need to be in the same location as your .tex file when you compile the document. Omit   \verb![width=.5in]! if you want the image to be full-sized.

\verb!\begin{figure}[ht]!\\
\verb!\includegraphics[width=.5in]{imagename.jpg}!\\
\verb!\caption{The (optional) caption goes here.}!\\
\verb!\end{figure}!

\subsection{Text decorations}

Your text can be \textit{italics} (\verb!\textit{italics}!), \textbf{boldface} (\verb!\textbf{boldface}!), or \underline{underlined} (\verb!\underline{underlined}!).

Your math can contain boldface, $\mathbf{R}$ (\verb!\mathbf{R}!), or blackboard bold, $\mathbb{R}$ (\verb!\mathbb{R}!). You may want to used these to express the sets of real numbers ($\mathbb{R}$ or $\mathbf{R}$), integers ($\mathbb{Z}$ or $\mathbf{Z}$), rational numbers ($\mathbb{Q}$ or $\mathbf{Q}$), and natural numbers ($\mathbb{N}$ or $\mathbf{N}$).

To have text appear in a math expression use \verb!\text!. \verb!(0,1]=\{x\in\mathbb{R}:x>0\text{ and }x\le 1\}! yields $(0,1]=\{x\in\mathbb{R}:x>0\text{ and }x\le 1\}$. (Without the \verb!\text! command it treats ``and'' as three variables: $(0,1]=\{x\in\mathbb{R}:x>0 and x\le 1\}$.)



\section{Spaces and new lines}

\LaTeX\ ignores extra spaces and new lines. For example, 

\verb!This   sentence will       look!

\verb!fine after      it is     compiled.!

This   sentence will       look
fine after      it is     compiled.


Leave one full empty line between two paragraphs. Place \verb!\\! at the end of a line to create a new line (but not create a new paragraph).

\verb!This!

\verb!compiles!

~

\verb!like\\!

\verb!this.!

This
compiles 

like\\
this.

Use  \verb!\noindent! to prevent a paragraph from indenting.

\section{Comments}

Use \verb!%! to create a comment. Nothing on the line after the \verb!%! will be typeset. \verb!$f(x)=\sin(x)$ %this is the sine function! yields $f(x)=\sin(x)$%this is the sine function

\section{Delimiters}

\begin{tabular}{lll}
\emph{description} & \emph{command} & \emph{output}\\
parentheses &\verb!(x)! & (x)\\
brackets &\verb![x]! & [x]\\
curly braces& \verb!\{x\}! & \{x\}\\
\end{tabular}

To make your delimiters large enough to fit the content, use them together with \verb!\right! and \verb!\left!. For example, \verb!\left\{\sin\left(\frac{1}{n}\right)\right\}_{n}^! \verb!{\infty}! produces\\ $\displaystyle \left\{\sin\left(\frac{1}{n}\right)\right\}_{n}^{\infty}$.

Curly braces are non-printing characters that are used to gather text that has more than one character. Observe the differences between the four expressions \verb!x^2!, \verb!x^{2}!, \verb!x^2t!, \verb!x^{2t}! when typeset: $x^2$, $x^{2}$, $x^2t$, $x^{2t}$.


\section{Lists}

You can produce ordered and unordered lists.

\begin{tabular}{lll}
\emph{description} & \emph{command} & \emph{output}\\
unordered list&
\begin{tabular}{l}
\verb!\begin{itemize}!\\
\verb!  \item!\\
\verb!  Thing 1!\\
\verb!  \item!\\
\verb!  Thing 2!\\
\verb!\end{itemize}!
\end{tabular}&
\begin{tabular}{l}
$\bullet$ Thing 1\\
$\bullet$ Thing 2
\end{tabular}\\
~\\
ordered list&
\begin{tabular}{l}
\verb!\begin{enumerate}!\\
\verb!  \item!\\
\verb!  Thing 1!\\
\verb!  \item!\\
\verb!  Thing 2!\\
\verb!\end{enumerate}!
\end{tabular}&
\begin{tabular}{l}
1.~Thing 1\\
2.~Thing 2
\end{tabular}
\end{tabular}


\section{Symbols (in \emph{math} mode)}

\subsection{The basics}
\begin{tabular}{lll}
\emph{description} & \emph{command} & \emph{output}\\
addition & \verb!+! & $+$\\
subtraction & \verb!-! & $-$\\
plus or minus & \verb!\pm! & $\pm$\\
multiplication (times) & \verb!\times! & $\times$\\
multiplication (dot) & \verb!\cdot! & $\cdot$\\
division symbol & \verb!\div! & $\div$\\
division (slash) & \verb!/! & $/$\\
circle plus & \verb!\oplus! & $\oplus$\\
circle times & \verb!\otimes! & $\otimes$\\
equal & \verb!=! & $=$\\
not equal & \verb!\ne! & $\ne$\\
less than & \verb!<! & $<$\\
greater than & \verb!>! & $>$\\
less than or equal to & \verb!\le! & $\le$\\
greater than or equal to & \verb!\ge! & $\ge$\\
approximately equal to & \verb!\approx! & $\approx$\\
infinity & \verb!\infty! & $\infty$\\
dots & \verb!1,2,3,\ldots! & $1,2,3,\ldots$\\
dots & \verb!1+2+3+\cdots! & $1+2+3+\cdots$\\
fraction & \verb!\frac{a}{b}! & $\frac{a}{b}$\\
square root & \verb!\sqrt{x}! & $\sqrt{x}$\\
$n$th root & \verb!\sqrt[n]{x}! & $\sqrt[n]{x}$\\
exponentiation & \verb!a^b! & $a^{b}$\\
subscript & \verb!a_b! & $a_{b}$\\
absolute value & \verb!|x|! & $|x|$\\
natural log  & \verb!\ln(x)! & $\ln(x)$\\
logarithms & \verb!\log_{a}b! & $\log_{a}b$\\
exponential function & \verb!e^x=\exp(x)! & $e^{x}=\exp(x)$\\
degree & \verb!\deg(f)! & $\deg(f)$\\
\end{tabular}
\newpage

\subsection{Functions}
\begin{tabular}{lll}
\emph{description} & \emph{command} & \emph{output}\\
maps to & \verb!\to! & $\to$\\
composition& \verb!\circ! & $\circ$\\
piecewise& \verb!|x|=! & \multirow{5}{*}{$\displaystyle |x|=\begin{cases}x&x\ge 0\\-x&x<0\end{cases}$}\\
function&\verb!\begin{cases}!&\\ 
&\verb!x & x\ge 0\\!&\\ 
&\verb!-x & x<0!&\\ 
&\verb!\end{cases}!&
\end{tabular}

\subsection{Greek and Hebrew letters}
\begin{tabular}{llll}
\emph{command} & \emph{output}&\emph{command} & \emph{output}\\
\verb!\alpha! & $\alpha$&\verb!\tau! & $\tau$\\
\verb!\beta! & $\beta$&\verb!\theta! & $\theta$\\
\verb!\chi! & $\chi$&\verb!\upsilon! & $\upsilon$\\
\verb!\delta! & $\delta$&\verb!\xi! & $\xi$\\
\verb!\epsilon! & $\epsilon$&\verb!\zeta! & $\zeta$\\
\verb!\varepsilon! & $\varepsilon$&\verb!\Delta! & $\Delta$\\
\verb!\eta! & $\eta$&\verb!\Gamma! & $\Gamma$\\
\verb!\gamma! & $\gamma$&\verb!\Lambda! & $\Lambda$\\
\verb!\iota! & $\iota$&\verb!\Omega! & $\Omega$\\
\verb!\kappa! & $\kappa$&\verb!\Phi! & $\Phi$\\
\verb!\lambda! & $\lambda$&\verb!\Pi! & $\Pi$\\
\verb!\mu! & $\mu$&\verb!\Psi! & $\Psi$\\
\verb!\nu! & $\nu$&\verb!\Sigma! & $\Sigma$\\
\verb!\omega! & $\omega$&\verb!\Theta! & $\Theta$\\
\verb!\phi! & $\phi$&\verb!\Upsilon! & $\Upsilon$\\
\verb!\varphi! & $\varphi$&\verb!\Xi! & $\Xi$\\
\verb!\pi! & $\pi$&\verb!\aleph! & $\aleph$\\
\verb!\psi! & $\psi$&\verb!\beth! & $\beth$\\
\verb!\rho! & $\rho$&\verb!\daleth! & $\daleth$\\
\verb!\sigma! & $\sigma$&\verb!\gimel! & $\gimel$
\end{tabular}


\subsection{Set theory}
\begin{tabular}{lll}
\emph{description} & \emph{command} & \emph{output}\\
set brackets & \verb!\{1,2,3\}! & $\{1,2,3\}$\\
element of & \verb!\in! & $\in$\\
not an element of & \verb!\not\in! & $\not\in$\\
subset of & \verb!\subset! & $\subset$\\
subset of & \verb!\subseteq! & $\subseteq$\\
not a subset of & \verb!\not\subset! & $\not\subset$\\
contains & \verb!\supset! & $\supset$\\
contains & \verb!\supseteq! & $\supseteq$\\
union & \verb!\cup! & $\cup$\\
intersection & \verb!\cap! & $\cap$\\
big union & 
\verb!\bigcup_{n=1}^{10}A_n! &
$\displaystyle \bigcup_{n=1}^{10}A_{n}$\\
big intersection & \verb!\bigcap_{n=1}^{10}A_n! &$\displaystyle \bigcap_{n=1}^{10}A_{n}$\\
empty set & \verb!\emptyset! & $\emptyset$\\
power set & \verb!\mathcal{P}! & $\mathcal{P}$\\
minimum & \verb!\min! & $\min$\\
maximum & \verb!\max! & $\max$\\
supremum & \verb!\sup! & $\sup$\\
infimum & \verb!\inf! & $\inf$\\
limit superior & \verb!\limsup! & $\limsup$\\
limit inferior & \verb!\liminf! & $\liminf$\\
closure & \verb!\overline{A}! & $\overline{A}$
\end{tabular}

\subsection{Calculus}
\begin{tabular}{lll}
\emph{description} & \emph{command} & \emph{output}\\
derivative & \verb!\frac{df}{dx}! & $\displaystyle \frac{df}{dx}$\\
derivative & \verb!\f'! & $f'$\\
partial derivative & 
\begin{tabular}{l}
\verb!\frac{\partial f}!\\ \verb!{\partial x}! 
\end{tabular}& $\displaystyle \frac{\partial f}{\partial x}$\\
integral & \verb!\int! & $\displaystyle\int$\\
double integral & \verb!\iint! & $\displaystyle\iint$\\
triple integral & \verb!\iiint! & $\displaystyle\iiint$\\
limits & \verb!\lim_{x\to \infty}! & $\displaystyle \lim_{x\to \infty}$\\
summation  & 
\verb!\sum_{n=1}^{\infty}a_n! &
$\displaystyle \sum_{n=1}^{\infty}a_n$\\
product  & 
\verb!\prod_{n=1}^{\infty}a_n! &
$\displaystyle \prod_{n=1}^{\infty}a_n$
\end{tabular}




\subsection{Logic}
\begin{tabular}{lll}
\emph{description} & \emph{command} & \emph{output}\\
not & \verb!\sim! & $\sim$\\
and & \verb!\land! & $\land$\\
or & \verb!\lor! & $\lor$\\
if...then & \verb!\to! & $\to$\\
if and only if & \verb!\leftrightarrow! & $\leftrightarrow$\\
logical equivalence & \verb!\equiv! & $\equiv$\\
therefore & \verb!\therefore! & $\therefore$\\
there exists  & \verb!\exists! & $\exists$\\
for all & \verb!\forall! & $\forall$\\
implies & \verb!\Rightarrow! & $\Rightarrow$\\
equivalent & \verb!\Leftrightarrow! & $\Leftrightarrow$
\end{tabular}

\subsection{Linear algebra}
\begin{tabular}{lll}
\emph{description} & \emph{command} & \emph{output}\\
vector & \verb!\vec{v}! & $\vec{v}$\\
vector & \verb!\mathbf{v}! & $\mathbf{v}$\\
norm & \verb!||\vec{v}||! & $||\vec{v}||$\\
matrix&
\begin{tabular}{l}
\verb!\left[!\\
\verb!\begin{array}{ccc}!\\
\verb!1 & 2 & 3 \\!\\
\verb!4 & 5 & 6\\!\\
\verb!7 & 8 & 0!\\
\verb!\end{array}!\\
\verb!\right]!\end{tabular}&
$\displaystyle \left[\begin{array}{ccc}1 & 2 & 3 \\4 & 5 & 6 \\7 & 8 & 0\end{array}\right]$\\
\\determinant&
\begin{tabular}{l}
\verb!\left|!\\
\verb!\begin{array}{ccc}!\\
\verb!1 & 2 & 3 \\!\\
\verb!4 & 5 & 6 \\!\\
\verb!7 & 8 & 0!\\
\verb!\end{array}!\\
\verb!\right|!
\end{tabular}&
$\displaystyle \left|\begin{array}{ccc}1 & 2 & 3 \\4 & 5 & 6 \\7 & 8 & 0\end{array}\right|$\\
determinant & \verb!\det(A)! & $ \det(A)$\\
trace & \verb!\operatorname{tr}(A)! & $\operatorname{tr}(A)$\\
dimension & \verb!\dim(V)! & $\dim(V)$\\
\end{tabular}

\subsection{Number theory}
\begin{tabular}{lll}
\emph{description} & \emph{command} & \emph{output}\\
divides & \verb!|! & $|$\\
does not divide & \verb!\not |! & $\not |$\\
div & \verb!\operatorname{div}! & $\operatorname{div}$\\
mod & \verb!\mod! & $\operatorname{mod}$\\
greatest common divisor & \verb!\gcd! & $\gcd$\\
ceiling & \verb!\lceil x \rceil! & $\lceil x\rceil$\\
floor & \verb!\lfloor x \rfloor! & $\lfloor x \rfloor$\\
\end{tabular}




\subsection{Geometry and trigonometry}
\begin{tabular}{lll}
\emph{description} & \emph{command} & \emph{output}\\
angle& \verb!\angle ABC! & $\angle ABC$\\
degree& \verb!90^{\circ}! & $90^{\circ}$\\
triangle& \verb!\triangle ABC! & $\triangle ABC$\\
segment& \verb!\overline{AB}! & $\overline{AB}$\\
sine& \verb!\sin! & $\sin$\\
cosine& \verb!\cos! & $\cos$\\
tangent& \verb!\tan! & $\tan$\\
cotangent& \verb!\cot! & $\cot$\\
secant& \verb!\sec! & $\sec$\\
cosecant& \verb!\csc! & $\csc$\\
inverse sine& \verb!\arcsin! & $\arcsin$\\
inverse cosine& \verb!\arccos! & $\arccos$\\
inverse tangent& \verb!\arctan! & $\arctan$\\
\end{tabular}

\section{Symbols (in \emph{text} mode)}

The followign symbols do \textbf{not} have to be surrounded by dollar signs.

\begin{tabular}{lll}
\emph{description} & \emph{command} & \emph{output}\\
dollar sign & \verb!\$! & \$ \\
percent & \verb!\%! & \% \\
ampersand & \verb!\&! & \& \\
pound & \verb!\#! & \# \\
backslash & \verb!\textbackslash! & \textbackslash \\
left quote marks & \verb!``! & `` \\
right quote marks & \verb!''! & '' \\
single left quote  & \verb!`! & ` \\
single right quote  & \verb!'! & ' \\
hyphen & \verb!X-ray! & X-ray\\
en-dash & \verb!pp. 5--15! & pp. 5--15 \\
em-dash & \verb!Yes---or no?! & Yes---or no? 
\end{tabular}

\section{Resources}
Great symbol look-up site: \href{http://detexify.kirelabs.org/}{Detexify}\\
\href{http://amath.colorado.edu/documentation/LaTeX/Symbols.pdf}{\LaTeX\ Mathematical Symbols}\\
\href{ftp://tug.ctan.org/pub/tex-archive/info/symbols/comprehensive/symbols-letter.pdf}{The Comprehensive \LaTeX\ Symbol List}\\ 
\href{http://mirrors.med.harvard.edu/ctan/info/lshort/english/lshort.pdf}{The Not So Short Introduction to \LaTeX\ 2$\varepsilon$}\\
\href{http://www.tug.org/}{TUG: The \TeX\ Users Group}\\
\href{http://www.ctan.org/}{CTAN: The Comprehensive \TeX\ Archive Network}\\
~\\
\LaTeX\ for the Mac: \href{http://www.tug.org/mactex/}{Mac\TeX}\\
\LaTeX\ for the PC: \href{http://www.texniccenter.org/}{{\TeX}nicCenter} and \href{http://miktex.org/}{MiK\TeX}\\
\LaTeX\ online: \href{http://www.writelatex.com/}{WriteLaTeX}.
\vfill
\hrule
~\\
Dave Richeson, Dickinson College, \href{http://divisbyzero.com/}{http://divisbyzero.com/}
\end{multicols}


\end{document}


